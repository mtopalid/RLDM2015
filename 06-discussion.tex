\section{Conclusion}

The aim of this model is to gain a better understanding of the role of the
basal ganglia in the formation of habits. It is based on a previous model by
\citet{Guthrie:2013} that explain the dynamic of action selection in the BG.
The model has been further refined with connections at the cortical level which
are consistent with neuro-anatomy. We also implemented Hebbian learning at the
cortical level, independently of reward. However, since BG helped to choose the
best action anytime, this results in having cortical learning to be naturally
modulated according to the value of the different cue, simply because the best
cue is chosen more often. After learning, the cortex is able to choose the best
cue without help of the BG, hence forming a new habit.


%% . Next step is to
%% include one more pathway, the indirect, known in bibliography of basal
%% ganglia. For this to happen, it must be added another nucleus of BG, Globus
%% Pallidus external (GPe), which takes part to this new pathway. In this way, we
%% can investigate the role of these three pathways and their interaction in
%% decision making. Our goal is the model to capture the essential features of the
%% biological system. In generality, computational models are used to frame
%% hypotheses that can be directly tested by biological experiments. Thus, through
%% the model we envy to participate in research around the complex functions of
%% decision making, learning and habitual behaviour.
