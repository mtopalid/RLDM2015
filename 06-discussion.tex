\section{Conlcusions}

	The aim of this model is to gain an understanding of the basal ganglia activity during procedural learning and habits. It is based on a previous model by Guthrie et al., 2013, and evolved in a way that it respects the new evidence for basal ganglia involvement to formatting of habits and learning. Next step is to include one more pathway, the indirect, known in bibliography of basal ganglia. For this to happen, it must be added another nucleus of BG, Globus Pallidus external (GPe), which takes part to this new pathway. In this way, we can investigate the role of these three pathways and their interaction in decision making. Our goal is the model to capture the essential features of the biological system. In generality, computational models are used to frame hypotheses that can be directly tested by biological experiments. Thus, through the model we envy to participate in research around the complex functions of decision making, learning and habitual behaviour.