\section{Description of the model}

Guthrie et al., 2013 presented a biophysically based, model of action selection that can solve the two-armed bandit task. Minimum details of brain circuitry was used. The model comprises two action selection modules: one for solving the cognitive action selection, and the other for solving the motor action selection. Each module consists of one instance of the center-surround architecture of Mink (1996), and the two modules are considered to be parallel, with inputs from distinct areas of cortex. In the model, when discussing the separation of action selection into two levels, we define a level as being a loop in which an area of cortex is in closed-loop feedback through the basal ganglia with itself (a cortico-basal ganglia or CBG loop) and channels as separate ensembles within that cortical area, each represent- ing a possible decision choice, that are in competition with one another during action selection.
The model instantiation is based on that of Leblois et al. (2006) that showed selection between two channels in one CBG loop. The module of action selection consists of a segregated, positive feedback direct pathway and a negative feedback hyperdirect pathway that is widely divergent from the subthalamic nucleus (STN) to GPi. Action selection is an intrinsic property of the interaction between the direct and hyperdirect pathways.
Here, the symmetry breaking is caused by introduced synaptic noise. Each CBG loop comprises an identical action selection module. To solve the task correctly, the decision made at the cognitive level must be available to guide the decision at the motor level. The information transfer between the two levels occurs at the level of the striatum (Fig. 1). The model also incorporates learning, modulated by a simulated dopamine reward signal to form a type of actor-critic network

	The architectural basis of the model has been originally described by Guthrie et al., 2013. The general architectural diagram of the formulation is illustrated in Figure \ref{fig:architecture}. In the original model, inactivation of basal ganglia output results inability of the network to produce a decision. To overcome this problem, connections at cortical level are added. 
	First, self-connections added to cortical nuclei. Each neuron inhibits all the others and excites itself. The idea was to enhance the competition between neurons.
	Then, the need for cross talking between the cognitive and motor part of cortex impelled us to include connectivity among the different cortical parts. To the previous model that happened only through basal ganglia. Thus, inactivation of GPi automatically means loss of this ability. 
	Yet,  Guthrie model did not provide time to cognitive cortex for making a decision, before motor  decides the appropriate move. They are a lot of cases that the two decisions are simultaneously, even the motor before the cognitive, at some cases. In order to obtain this difference in timing, except the added connections, a part of the original parameters have changed. 
	One of the results from experiments by Piron et al., 2015, was the success of monkeys to reserve their knowledge after inactivation of GPi. This is a strong indication for storage of information outside of basal ganglia. It's known that striatal learning follows the rules of reinforcement learning. Each time BG computes the expected reward of an action and adjusts the weights according to the error of the real one. In the opposite side is the cortical learning. Here, learning occurs after every move without the need of any computation or outcome. Therefore, hebbian learning is enough. Basal ganglia won't let wrong cues to be chosen a lot of times, so because of the statistical appearance of the chosen cues, cortex will learn accurately.  The statistics also, is the explanation for the absence of learning when basal ganglia don't contribute to the decision; the choices at this point will be random.