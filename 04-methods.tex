\section{Description of the model}

	The architectural basis of the model has been originally described by Guthrie et al., 2013. The general architectural diagram of the formulation is illustrated in Figure \ref{fig:architecture}. In the original model, inactivation of basal ganglia output results inability of the network to produce a decision. To overcome this problem, connections at cortical level are added. 
	First, self-connections added to cortical nuclei. Each neuron inhibits all the others and excites itself. The idea was to enhance the competition between neurons.
	Then, the need for cross talking between the cognitive and motor part of cortex impelled us to include connectivity among the different cortical parts. To the previous model that happened only through basal ganglia. Thus, inactivation of GPi automatically means loss of this ability. 
	Yet,  Guthrie model did not provide time to cognitive cortex for making a decision, before motor  decides the appropriate move. They are a lot of cases that the two decisions are simultaneously, even the motor before the cognitive, at some cases. In order to obtain this difference in timing, except the added connections, a part of the original parameters have changed. 
	One of the results from experiments by Piron et al., 2015, was the success of monkeys to reserve their knowledge after inactivation of GPi. This is a strong indication for storage of information outside of basal ganglia. It's known that striatal learning follows the rules of reinforcement learning. Each time BG computes the expected reward of an action and adjusts the weights according to the error of the real one. In the opposite side is the cortical learning. Here, learning occurs after every move without the need of any computation or outcome. Therefore, hebbian learning is enough. Basal ganglia won't let wrong cues to be chosen a lot of times, so because of the statistical appearance of the chosen cues, cortex will learn accurately.  The statistics also, is the explanation for the absence of learning when basal ganglia don't contribute to the decision; the choices at this point will be random.