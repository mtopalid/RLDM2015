\section{Methods}

The architectural basis of the model has been originally described in
\citet{Guthrie:2013} where authors introduced a biophysical model of action
selection that can solve a two-arm bandit task. Two parallel action selection
pathways compose the model with inputs from distinct areas of the cortex: one
for handling the cognitive action selection, and the other for the motor
selection. The model includes the cortex (Cx), the thalamus (Th) and several
nuclei of the BG: striatum (Str), the subthalamic nuclei (STN) and the globus
pallidus (GPi). Each module is made of a closed-loop positive feedback direct
pathway (Cx-Str-GPi-Th-Cx) and a closed-loop negative feedback hyperdirect
pathway (Cx-STN-GPi-Th-Cx) and is based on the center-surround architecture of
\cite{Mink:1996}. The interactions between these two pathways are able to
induce an action selection at the motor level. However, the task requires first
the actual selection of the best cue before performing the corresponding motor
action. In the \citet{Guthrie:2013} model, this is implemented at the striatal
level where a dopamine reward signal is used to implement a simple value-based
learning. The general architectural of the model is illustrated on Figure
\ref{fig:architecture}.

However, in the original model, the inactivation of the basal ganglia output
(GPi) results in the inability of the model to make a decision since there is
no competitive mechanism at the cortical level. We thus added lateral
connections in each cortical modules (self-excitation and surround inhibition)
such that a unique cognitive and motor decision can be made. At this point,
there is no guarantee that the motor decision corresponds to the cognitive
ones. The model can choose a cue A but moves toward the location of a cue B. To
overcome this problem, we also need to establish a cross-talking between the
different cortical modules, independently of the BG pathways. This has been
made using excitatory connections from and to the associative cortical
module. With proper Hebbian learning, this allows the cortex to make a
consistent decision in the absence of GPi, even if it does not guarantee to
make the optimal decision.

One important property of the cortical decision is that it is significantly
slower than the BG decision. This can be shown by measuring the time of motor
decision, which has been defined at the time where the difference between the
two most activated units in the motor cortex is greater than a given threshold
(40Hz). Before learning, the BG decision time (GPi is intact) is around 250ms
while the cortical decision time (GPi is disabled) is around 800 ms. This
difference in timing is actually critical for the BG to teach the cortex as
explained in the results section.
