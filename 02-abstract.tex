\begin{abstract}
If basal ganglia are widely accepted to participate in the high-level cognitive function of decision-making, their role is less clear regarding the formation of habits. One of the biggest problem is to understand how goal-directed actions are transformed into habitual responses, or, said differently, how an animal can shift from an action-outcome (A-O) system to a stimulus-response (S-R) one while keeping a consistent behavior.

We introduce a computational model (basal ganglia, thalamus and cortex) that can solve a simple two arm-bandit task using reinforcement learning and explicit valuation of the outcome (\citet{Guthrie2013}). Hebbian learning has been added at the cortical level such that the model learns each time a move is issued, rewarded or not. Then, by inhibiting the output nuclei of the model (GPi), we show how learning has been transfered from the basal ganglia to the cortex, simply as a consequence of the statistics of the choice. Because best (in the sense of most rewarded) actions are chosen more often, this directly impacts the amount of Hebbian learning and lead to the formation of habits within the cortex.

These results have been confirmed in monkeys (unpublished data at the time of writing) doing the same tasks where the BG has been inactivated using muscimol. This tends to show that the basal ganglia implicitely teach the cortex in order for it to learn the values of new options. In the end, the cortex is able to solve the task perfectly, even if it exhibits slower reaction times.
\end{abstract}