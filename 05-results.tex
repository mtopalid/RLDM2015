\section{Results}

The main task at the experiments by \citet{Piron2015} was a two-armed bandit
paradigm. During an exploratory phase by trial and errors, the monkeys appraise
what are the outcomes (various probability of reward in most of the
cases). Afterwards, they choose preferentially the best option (maximizes the
reward). Most of the species don't have the ability to conclude the outcome,
and for those that do have it (primates amongst them), it takes more than one
session of training.

At the experiments, first, only two cues were always presented to the monkey,
with probabilities of reward 0.75 and 0.25. After the learning phase, a testing
one followed. Now, a pair of cues of which they know the values or a new set of
cues of unknown values was presented. By pharmacological approach, they tested
with and without inactivation of the major output of basal ganglia, Globus
Pallidus (GPi), toward the cortex through the thalamus in order to unravel
their role in decision making and learning processes.

For testing our model, we followed the protocol by \citet{Piron2015}. There
were 50 trials in the learning phase in order the network to find out the
outcome of two cues. Followed by testing, which comprises of 120 trials
presenting a set of cues with known values and 120 with a pair of unknown, with
or without active GPi.

Our results were in accordance with the ones of the experiments. Figure
\ref{fig:Performances} shows that the network maximizes its choice at known
values cues in both cases of GPi, with mean success of $100.00\pm 0.0\%$. Also,
the choice was faster with the contribution of basal ganglia that otherwise
(Figure \ref{fig:RTresults}). When cues with unknown values were presented with
active GPi, at the beginning the selection is random until finally display a
preference for the target associated to the best utility (mean $54.9 \pm 5.1\%$
to $98.9 \pm 0.4\%$). On the other hand, without the help of GPi, it was not
able to display any preference for any of the two targets (mean $49.6 \pm
6.2\%$ to $59.4\pm 3.8\%$) and more time was needed to make a decision (Figure
\ref{fig:RTresults}).
