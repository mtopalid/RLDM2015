\section{Results}

Following the protocol of experiments by \citet{Piron2015}, we
performed 250 simulations of 480 trials each. In Habitual Condition 
(HC), the two targets presented to the model are the ones with which 
the system was trained. In Novelty Condition (NC), the targets 
presented are new. The first 120 trials were dedicated in normal 
conditions and the network was exposed to familiar cues, and 120 
trials for unfamiliar. Then, we inactivated GPi, and tested the system to the two cases with same amount of trials. As optimum trial is defined the number of trials in which the model chose the target associated to the best utility, normalized by the number of trials. Further, the reaction time of the network is the time needed a move to occur from the presentation of the cues.
       
Our results are in accordance with the experiments in monkeys.
When all units of the model were participating in the choice 
in HC, the proportion of optimum trials was  
$100.00\pm 0.0\%$ (Figure \ref{fig:Performances}). In NC, 
the choice was random at the beginning of training 
($52.5 \pm 0.0013\%$, Figure \ref{fig:Performances}) to finally 
express a preference for the optimum target ( $93.0 \pm 0.0\%$, 
Figure \ref{fig:Performances}). In the second condition, the mean 
reaction time is significantly higher than in the first (respectively  
$190.8 \pm 27.71ms$ and $143.26 \pm 0.616ms$, Figure
\ref{fig:RTresults}).

Inactivation of the GPi, does not impair habitual behaviour. The 
ratio of optimum trials does not vary significantly ( $100.00\pm 
0.0\%$ , Figure \ref{fig:Performances}) as compared to the 
previous condition. On the other hand, when unknown cues were
given, the system was not able to display any preference for any of 
the two targets ( $50.3 \pm 0.04\%$ to  $52.6\pm 0.0004\%$,
Figure \ref{fig:Performances}). The absence of GPi significantly 
increased the reaction time for both cases, with familiar cues
($416.04 \pm 0.88ms$, Figure\ref{fig:RTresults}) and unfamiliar 
($853.76 \pm 37.22ms$, Figure\ref{fig:RTresults}) as compared to 
the normal condition.
