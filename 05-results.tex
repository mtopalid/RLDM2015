\section{Results}
We tested the model using 4 differents paradigms:
\begin{itemize}
\item HC/GPi:   Habitual condition using alrealy learned stimuli with intact GPi.
\item HC/NoGPi: Habitual condition using alrealy learned stimuli with lesioned GPi.
\item NC/GPi:   Novel condition using non familar stimuli with intact GPi.
\item NC/NoGPi: Novel condition using non familar stimuli with lesioned GPi.
\end{itemize}
Each condition has been tested for 250 experiments where each experiment
consists in 120 consecutive trials (presentation of the cues, decision,
potential reward and reset). Before starting an new experiment, the model is
trained on the familiar stimuli until the performance is over 0.95. We measured
performances as the ratio of optimal choices compared to the number of
trials. Response time has been recorded as the time of the motor decision, that
is, when the difference between two greatest motor activation is greater than
the decision threshold (40Hz). This time is relative to the stimulus onset
(t=500ms).

Our results are in accordance with the experiments in monkeys. In the habitual
condition, performances are optimal with or without lesion, indicating the
cortex is able to make the optimal decision without the help of the basal
ganglia. However, in the novel condition, things are quite different. For the
intact model, the model starts a trial at chance level, giving random
choices. However, after a few trials (around 15), it reaches a near-optimal
performance, indicating the model has learned the respective reward proability
associated with each cue. For the lesioned model, the performances stay at the
level of chance, indicating the cortes is unable ot learn the task ``on its
own''. It is to be noted that due ot Hebbian learning, the lesioned model tend
to first choose a given random cue and stick to this choice until the end of
the experiment. If this is the right cue, the performance can reach 1 for a
single experiment, but over the course of the 250 experiments, the mean
performance is around 0.5.


%% When all units
%% of the model were participating in the choice in HC, the proportion of optimum
%% trials was $100.00\pm 0.0\%$ (Figure \ref{fig:Performances}). In NC, the choice
%% was random at the beginning of training ($52.5 \pm 0.0013\%$, Figure
%% \ref{fig:Performances}) to finally express a preference for the optimum target
%% ( $93.0 \pm 0.0\%$, Figure \ref{fig:Performances}). In the second condition,
%% the mean reaction time is significantly higher than in the first (respectively
%% $190.8 \pm 27.71ms$ and $143.26 \pm 0.616ms$, Figure \ref{fig:RTresults}).

%% Inactivation of the GPi, does not impair habitual behaviour. The ratio of
%% optimum trials does not vary significantly ( $100.00\pm 0.0\%$ , Figure
%% \ref{fig:Performances}) as compared to the previous condition. On the other
%% hand, when unknown cues were given, the system was not able to display any
%% preference for any of the two targets ( $50.3 \pm 0.04\%$ to $52.6\pm
%% 0.0004\%$, Figure \ref{fig:Performances}). The absence of GPi significantly
%% increased the reaction time for both cases, with familiar cues ($416.04 \pm
%% 0.88ms$, Figure\ref{fig:RTresults}) and unfamiliar ($853.76 \pm 37.22ms$,
%% Figure\ref{fig:RTresults}) as compared to the normal condition.
